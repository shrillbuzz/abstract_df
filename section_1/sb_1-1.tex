\documentclass{article}

% Language setting
% Replace `english' with e.g. `spanish' to change the document language
\usepackage[english]{babel}
% Set page size and margins
% Replace `letterpaper' with`a4paper' for UK/EU standard size
\usepackage[letterpaper,top=2cm,bottom=2cm,left=3cm,right=3cm,marginparwidth=1.75cm]{geometry}

% Useful packages
\usepackage{amsmath}
\usepackage{amsfonts}
\usepackage{graphicx}
\usepackage[colorlinks=true, allcolors=blue]{hyperref}


\newcommand{\Z}{\mathbb{Z}}
\newcommand{\residue}{\mathbb{Z}/n\mathbb{Z}}



\begin{document}


% 3, 4, 5

% 1, 3, 4, 5, 9, 15, 16, 17, 19, 20, 21, 23, 25, 26, 27, 29, 30, 31, 32, 33, 34, 35

\section*{3 \& 4}

Notice that for any integers x, y the following properties holds for their residue classes mod n:
\[\overline{x * y} = \bar x * \bar y \quad \quad \overline{x+y} = \bar x + \bar y\]

The associativity of multiplication and addition of residue classes in  \(\residue\) follows from these facts and the associativity of multiplication and addition over the integers. Let \(a, b, c \in \Z\).

\begin{align*}
    (\bar x + \bar y) + \bar z &=  \overline{(x+y) + z}\\
    &= \overline{x+ (y+z)}\\
    &= \bar x + (\bar y + \bar z)
\end{align*}

The argument is essentially the same for multiplication.

\section*{5}

 Let \(n > 1\). Now consider \(\bar 0\in \residue\). It's not possible for \(\bar 0\) to have a multiplicative inverse, because any value multiplied by \(\bar 0\) is \(\bar 0\).  Thus \(\residue\) for n greater than one are not groups.
 

\section*{7}
Define for any real number x:
\[\bar x = x - [x]\]

Notice that for any \(x, y \in G\) 
\[x \star y = x + y - [x +y] = \overline{x+y}\]
The associativity and commutativity of \(\star\) follow from this definition (combined with the associativity and commutativity of addition). 
\[x \star y = \bar x \star \bar y = \overline{x+y}\]
\begin{itemize}
    \item well defined.

    If \(x = x'\) and \(y = y'\), then 
    \[x \star y = x + y - [x + y] = x' + y' - [x' + y'] = x' \star y'\]

  \item binary operation/closure

\(x \star y\) is at least 0 (when x+y is an integer) and no more than one. Thus \(x \star y \in G\) 


    \item associative

    \[(x \star y) \star z = \overline{(x+y) +z} = \overline{x+(y +z)} = x \star (y \star z) \]

    \item commutative
    \[x \star y = \overline{x+y} = \overline{y + x} = y \star x\]
    
    \item identity

    For any x in G:
    \[x \star 0 = \overline{x+0} = x\]
    
    \item every element has an inverse

    Let x be an arbitrary element of G. Then 1-x is its inverse:
    \[x \star (1-x) = \overline{x + 1 - x} = \bar 1 = 0\]
    
    
\end{itemize}



\section*{9}

\begin{itemize}
    \item (a) ...


\item (b)

\begin{itemize}
    \item associativity, binary operation ...
    \item identity element
    
    For any x in G, x*1 = x so 1 is identity
    \item everything has inverse 

    Let \(a + b\sqrt{2} \in G\). We want to show some \(x, y \in \mathbb Q\) exist such that
    \[(a + b\sqrt{2})(x + y\sqrt{2}) = ax + b\sqrt{2}y\sqrt{2} + ay\sqrt{2} +b\sqrt{2}ax  = ax + 2by + (ay + bax)\sqrt{2} =  1\]

    For this to be the case, the following system of equations must be satisfiable:

    \[ax + 2by = 1 \quad \quad  ay + bax = 0\]

    ... (this actually seems pretty hard because for some values of a, b it seems line there is no solution?)

\end{itemize}
\end{itemize}


\section*{15}

Very straight forwar/. I'll do this once I can type ion my laptop and not have to deal with the evilness of qwerty. 



\section*{16}

First direction: If the order of an element of a group is 2 then by definition, that element squared is 1. If the order is 1, then that element to any power including 2 is 1. 

Second direction: Suppose x doesn't have order 1 or 2 (in other words x has order greater than 2). Thus the smallest positive integer n such that \(x^n =1\) is greater than 2. It can't be the case then that  \(x^2 = 1\).

\section*{17}

\section*{19}

\section*{20}



% \section{P 2.1.6}

% \subparagraph{ii} addition is not associative
% $$((a,b) + (c,d)) + (e,f) = (b+d, a+c) + (e,f)$$
% $$ (b+d, a+c) + (e,f)= (a+c+f,b+d+e)$$

% $$(a,b) + ((c,d) + (e,f)) = (a,b) + (d+f,c+e) $$
% $$(a,b) + (d+f,c+e) = (c+e+b,d+f+a)$$

% \section{P 2.2.1} Determine which are subspaces of \(\mathbb{R}^3\)
% \paragraph{a} All vectors of the form (a,b,c), where b = 2a+c are a subspace

% \subparagraph{i} addition is closed

% $$(a_1,2a_1+c_1,c_1)+(a_2,2a_2+c_2,c_2) = (a_1+a_2,2(a_2+a_1)+(c_1 +c_2) ,c_1+c_1)$$

% \subparagraph{ii} • is closed

% $$\alpha (a,b,c) = (\alpha a, 2(\alpha a)+\alpha c,\alpha c)$$

% \subparagraph{iii} has zero vector of \(\mathbb{R}^3\)

% $$(0,0,0) \in(a,2a+c, c)$$

% \paragraph{b} All vectors of the form (a,b,2) are not a subspace
% \subparagraph{i} Not closed under • 

% $$2*(a,b,2) = (2a,2b,4) \not\in (a,b,2) $$


% %•••••••••%
% \section{P 2.2.5} Which of the following functions are subspaces of V? Prove your assertion. 
% %•••••••••%
% \paragraph{(a)} $f \in V$ for which f(1) = 0, is a subspace

% \subparagraph{i} addition is closed
% $$(g + f)(x) = g(x) + f(x) $$
% $$g(1) + f(1) = 0$$
% $$(g + f)(x) \in V$$


% \subparagraph{ii} • is closed

% $$\alpha f(1) = 0$$
% $$\alpha f(x) \in V$$


% \subparagraph{iii} has zero vector of V
% $$f(x) = 0 $$

% $$g(x) + f(x) = g(x)$$


% \paragraph{(b)}$f \in V$ for which f(1) = 1,  is not a subspace

% \subparagraph{i} addition is not closed

% $$(g + f)(x) = f(x) +g(x)  $$
% $$ (g + f)(1) = g(1) + f(1) = 2$$
% $$(g + f)(x) \mbox{ is not in the subspace}$$



\end{document}