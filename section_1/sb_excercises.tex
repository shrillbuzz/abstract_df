\documentclass{article}
\title{Part I - Group Theory}

% Language setting
% Replace `english' with e.g. `spanish' to change the document language
\usepackage[english]{babel}
% Set page size and margins
% Replace `letterpaper' with`a4paper' for UK/EU standard size
\usepackage[letterpaper,top=2cm,bottom=2cm,left=3cm,right=3cm,marginparwidth=1.75cm]{geometry}

% Useful packages
\usepackage{amsmath}
\usepackage{amsfonts}
\usepackage{graphicx}
\usepackage[colorlinks=true, allcolors=blue]{hyperref}

\setcounter{tocdepth}{2} 
\usepackage{etoc}


\newcommand{\Z}{\mathbb{Z}}
\newcommand{\F}{\mathbb{F}}
\newcommand{\R}{\mathbb{R}}
\newcommand{\residue}{\mathbb{Z}/n\mathbb{Z}}
\newcommand{\prob}[1]{\section*{#1}}


\newcommand{\problemslocaltoc}{%
  \etocsettocstyle{}{}%
  \etocsetnexttocdepth{subsubsection}%
%   \setcounter{secnumdepth}{2}%
  \localtableofcontents%
}


\newcounter{problem}
\renewcommand{\thesubsubsection}{\thesubsection.\arabic{problem}}

\newcommand{\problem}[1]{%
  \setcounter{problem}{#1}\addtocounter{problem}{-1}%
  \stepcounter{problem}%
  \subsubsection{}%
}


\begin{document}
\tableofcontents

\section{Introduction to Groups}
\subsection{Basic Axioms and Examples}

% \etocsettocstyle{}{} % no title
% \etocsetnexttocdepth{subsubsection} % adjust depth if needed
% \setcounter{secnumdepth}{2} % don't number subsubsections
% \localtableofcontents
\problemslocaltoc

% 3, 4, 5

% 1, 3, 4, 5, 9, 15, 16, 17, 19, 20, 21, 23, 25, 26, 27, 29, 30, 31, 32, 33, 34, 35

% \problem{3 \& 4}

Notice that for any integers x, y the following properties holds for their residue classes mod n:
\[\overline{x * y} = \bar x * \bar y \quad \quad \overline{x+y} = \bar x + \bar y\]

The associativity of multiplication and addition of residue classes in  \(\residue\) follows from these facts and the associativity of multiplication and addition over the integers. Let \(a, b, c \in \Z\).

\begin{align*}
    (\bar x + \bar y) + \bar z &=  \overline{(x+y) + z}\\
    &= \overline{x+ (y+z)}\\
    &= \bar x + (\bar y + \bar z)
\end{align*}

The argument is essentially the same for multiplication.

\problem{5}

 Let \(n > 1\). Now consider \(\bar 0\in \residue\). It's not possible for \(\bar 0\) to have a multiplicative inverse, because any value multiplied by \(\bar 0\) is \(\bar 0\).  Thus \(\residue\) for n greater than one are not groups.
 

\problem{7}
Define for any real number x:
\[\bar x = x - [x]\]

Notice that for any \(x, y \in G\) 
\[x \star y = x + y - [x +y] = \overline{x+y}\]
The associativity and commutativity of \(\star\) follow from this definition (combined with the associativity and commutativity of addition). 
\[x \star y = \bar x \star \bar y = \overline{x+y}\]
\begin{itemize}
    \item well defined.

    If \(x = x'\) and \(y = y'\), then 
    \[x \star y = x + y - [x + y] = x' + y' - [x' + y'] = x' \star y'\]

  \item binary operation/closure

\(x \star y\) is at least 0 (when x+y is an integer) and no more than one. Thus \(x \star y \in G\) 


    \item associative

    \[(x \star y) \star z = \overline{(x+y) +z} = \overline{x+(y +z)} = x \star (y \star z) \]

    \item commutative
    \[x \star y = \overline{x+y} = \overline{y + x} = y \star x\]
    
    \item identity

    For any x in G:
    \[x \star 0 = \overline{x+0} = x\]
    
    \item every element has an inverse

    Let x be an arbitrary element of G. Then 1-x is its inverse:
    \[x \star (1-x) = \overline{x + 1 - x} = \bar 1 = 0\]
    
    
\end{itemize}



\section*{9}

\begin{itemize}
    \item (a) ...


\item (b)

\begin{itemize}
    \item associativity, binary operation ...
    \item identity element
    
    For any x in G, x*1 = x so 1 is identity
    \item everything has inverse 

    Let \(a + b\sqrt{2} \in G\). We want to show some \(x, y \in \mathbb Q\) exist such that
    \[(a + b\sqrt{2})(x + y\sqrt{2}) = ax + b\sqrt{2}y\sqrt{2} + ay\sqrt{2} +b\sqrt{2}ax  = ax + 2by + (ay + bax)\sqrt{2} =  1\]

    For this to be the case, the following system of equations must be satisfiable:

    \[ax + 2by = 1 \quad \quad  ay + bax = 0\]

    ... (this actually seems pretty hard because for some values of a, b it seems line there is no solution?)

\end{itemize}
\end{itemize}


\problem{15}

We can show 

Notice:
\begin{align*}
    \left ( a_1 a_2 \cdots a_n \right ) \cdot \left ( a_n^{-1} a_{n-1}^{-1}\cdots a_1^{-1} \right ) &=\left ( a_1 a_2 \cdots a_{n-1} \right ) a_n \cdot  a_n^{-1} \left ( a_{n-1}^{-1}\cdots a_1^{-1} \right )\\
    &= \left ( a_1 a_2 \cdots a_{n-1} \right )  \cdot   \left ( a_{n-1}^{-1}\cdots a_1^{-1} \right ) \\
\end{align*}



\problem{16}

First direction: If the order of an element of a group is 2 then by definition, that element squared is 1. If the order is 1, then that element to any power including 2 is 1. 

Second direction: Suppose x doesn't have order 1 or 2 (in other words x has order greater than 2). Thus the smallest positive integer n such that \(x^n =1\) is greater than 2. It can't be the case then that  \(x^2 = 1\).

\problem{17}

\problem{19}

\problem{20}

\subsection{Matrix Groups}

\problemslocaltoc

\problem 1 

Prove that \( |GL_2(\mathbb{F}) |= 6 \)


\problem 2

Write out all the elements of \(GL_2 (F_2) \) and compute the order of each element.

\problem 3 Show that \( GL_2 ( \F  ) \) is non-abelian.

\problem 4

Show that if n is not prime then \(\residue\) is not a field. 

\problem 5 

Show that \(GL_n(F)\) is a finite group if and only if F has a finite number of elements


\problem 6

If \(|F| = q\) is finite prove that \(GL_n(F) < q^{n^2}\).

\problem 7 

Let p be a prime.
Prove that the order of \(GL_2 (\F_p) \) is \(p^4 -p^3 -p^2 + p\). 
(do not just quote the order formula in this section). 
[Subtract the number of 2 x 2 matrices over \(\F_p\). 
You may use tha fact that a 2 x 2 matricx is not invertible if and only if one row is a muliple of the other. ]

\problem 8

Show that \(GL_n(F)\) is non-abelian for any \(n \ge 2\) and any F. 


\problem 9 

Prove that the binary operation of matrix muliplication of 2 x 2 matricies iwth real entities is associative


\problem {10}
Let \(H(F) = \{\begin{bmatrix}
1 & a & b \\
0 & 1 & c \\
0 & 0 & 1
\end{bmatrix} | a, b, c, d \in F\}\) -- called the Heisenberg group over F. Let 
\(X = \begin{bmatrix}
    1 & a & b\\
    0 & 1 & c\\
    0 & 0 & 1
\end{bmatrix}\) and \(Y = \begin{bmatrix}
    1 & d & e\\
    0 & 1 & f \\
    0 & 0 & 1
\end{bmatrix}\) be elements of \(H(F)\).

\begin{enumerate}
    \item Compute the matrix product \(XY\) and deduce that \(H(F)\) is closed under matrix multiplication. 
    Exhibit explicit maricies such that \(XY  \not = Y X\) (so that \(H(F)\) is always non-abelian).
    \item Find and explicit formula for the matrix inverse \(X^{-1}\) and deduce that \(H(F)\) is closed under inverse.
    \item Prove the associative law for \(H(F)\) and deduce that \(H(F)\) is a group of order \(|F|^3\). 
    \item Find the order of each element of the finite group \(H(\Z / 2 \Z)\).
    \item Prove that every nonidentity elements of the group \(H (\R )\) has infinite order.
\end{enumerate}



\end{document}